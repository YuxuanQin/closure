\documentclass{ctexart}
\bibliographystyle{alpha}


%% \mathbb 的依赖
\usepackage{amssymb}

%% 用来实现交叉引用的宏包
\usepackage[dvipsnames]{xcolor}
\usepackage{hyperref}
\hypersetup{colorlinks=true, pdfstartview=FitV, linkcolor=BrickRed,citecolor=BrickRed, urlcolor=BrickRed}

% algorithm2e生成伪代码的包:ruled(三线表样式)、linesnumbered(显示行号)
\usepackage[ruled,linesnumbered]{algorithm2e}

% 本段代码使得各节标题左对齐,删掉可以居中。
\ctexset{
section = {
format = \raggedright\Large\bfseries,
}
}

\title{用算法求 Taffy 的真实身份}
\author{孙笑川\thanks{Email: foo@doma.in} \and 孙哭川\thanks{Email: bar@doma.in}}
\date{\today}



%%%%%%%%%%%%%%%%%%%%%%%%%%%%%%%%%%%%%%%%%%%%%%%%%%%%%%%%%%%%%%%%%%%%%%%%



%% 文章开始
\begin{document}
  
  \maketitle
  % 下面的 \cite{coffee} 是引用参考文献中名为 “coffee” 的那一篇
  \begin{abstract}
    {在这篇文章里,我们详细探讨了 taffy 的真面目,通过传递闭包算法,我们在前人的基础~\cite{coffee}上,终于找到了真相!}
    \par\textbf{关键词:} Taffy,算法。  
  \end{abstract}


  \tableofcontents

%% 引言
  \section{引言}
    据悉,``taffy" 是 ``coffee" 的谐音。  % 别忘了 `` 是前引号


%% 算法原理
  \section{算法原理}\label{code}  % 此处 \label{name} 是为了以后引用本节
    首先,让我们来观察如下代码:

    % 在 verbatim 之间随意书写你的代码就可以了。
    \begin{verbatim}
      #include <stdio.h>
      int main (void) {
        return main()
      }
    \end{verbatim}
    % \verb|...| 会将内部内容以无衬线字体显示,也就是编程字体。
    这段代码的精髓就在于,它完美地揭示了 \verb|main| 函数的真相——\verb|main| 函数被定义为返回一个 \verb|int| 型,那么,它要返回的 \verb|int| 是什么呢?答案就是,它要返回的那个数,就是它要返回的那个数!实在是太巧妙了!


%% 证明
  \section{证明}
    因为 

    \[
      \sum_{k \in \mathbb{N}} \frac{1}{k^2} = \frac{\pi ^2}{6}
    \]
   
    所以 taffy 就是 taffy!
    \subsection{公式分析}  % 这是子节
      这个公式是 Taffy 托梦给我的,所以不会错!


%% 时间复杂度
  \section{时间复杂度}
    在本节,我们将证明给出的算法(第 \ref{code} 节)时间复杂度为 \( O(\frac{1}{n}) \)。  % 这里的 \ref 就是引用了代码中所定义的 code 节
    但是在证明之前,我们必须指出:


%% 伪代码分析
  \section{伪代码分析}
    如下伪代码揭示了怎么用 coffee 寻找 Taffy:

  \begin{algorithm}[H]
  \SetAlgoLined
  
  \caption{Search Taffy}  % 算法名称

  \KwIn{ Coffee's name }  % 输入数据
  \KwOut{ Taffy's location }  % 输出数据
  \tcp{temp: the temperature function}  % \tcp 是伪代码内部的注释
  
  % \while{条件}{循环体}
  \While{coffee is not cold}{
  print (temp (coffee))\;  % \; 使行末添加分号,并自动换行
    
    % \eIf{条件}{肯定语句}{否定语句}
    \eIf{coffe is hot}{
	change a cup of coffee\;
      }{
	put coffee on the table\;
      }

    % \If{条件}{肯定语句}
    \If{Taffy is near table}{
      print (``Taffy is around you!")\;
      }
    }

  % \For{条件}{循环语句}
  \For{all coffee}{
    drink\;
  }

  return Nothing\;  % 返回值
  \end{algorithm}    

  我们的伪代码效率高达 1.00\%!


%% 附录
  \appendix
  \section{附录}
    这里是附录,这里没有 taffy。




%% 参考文献
  \begin{thebibliography}{10}
    \bibitem{coffee}  % 这是你引用文献时的简称
	    X. Sun: \emph{What is coffee}, Chouxiang Volume~9, Issue 2 (2024).
  \end{thebibliography}

\end{document}
